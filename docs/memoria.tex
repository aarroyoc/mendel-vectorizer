\documentclass{report}
\usepackage[utf8]{inputenc}
\usepackage[spanish]{babel}


\title{Vectorizado de siluetas por algoritmos genéticos}
\author{Adrián Arroyo Calle}
\date{ Octubre-Noviembre 2017}
\begin{document}

\maketitle

\chapter{Introducción}
En el mundo gráfico existen dos tipos fundamentales de imágenes. En primer lugar tenemos las rasterizadas, donde se almacena la información de cada píxel. Estas imágenes son las generadas por escáneres y cámaras fotográficas, así como por renders de ordenador. Los formatos más usados son JPEG y PNG. En segundo lugar tenemos las imágenes vectoriales, donde se almacenan funciones matemáticas. Este tipo de imágenes son muy usadas en ilustraciones, diseño gráfico y videojuegos. Los formatos más usados son SVG y AI. 

Para obtener una imagen rasterizada desde una imagen vectorial, el procedimiento es sencillo. Simplmente se aplican las fórmulas matemáticas sobre un lienzo de un determinado tamaño. En cambio, el procedimiento inverso es extremadamente complicado. 

El algoritmo genético presentado en este documento permite transformar imágenes rasterizadas sencillas en imágenes vectoriales.

\chapter{Estructura general}

El procedimiento básico que seguirá el programa, que será implementado en el lenguaje 
de programación Rust\cite{Matsakis:2014:RL:2692956.2663188}, será el siguiente:

\begin{enumerate}
	\item Leer imagen rasterizada de entrada
	\item Pasar la imagen a escala de grises
	\item Detección de esquinas en la imagen
	\item Optimización de las conexiones entre esquinas
	\item Salida de la imagen vectorial
\end{enumerate}

Para los 3 primeros pasos usaremos librerías de código abierto disponibles para Rust. Concretamente
usaremos image e imageproc. Esta última contiene los algoritmos de detección de esquinas FAST9 y 
FAST12.

Será en el paso 4 es donde se aplicará el algoritmo genético, ya que se trata de una tarea de 
optimización. Los algoritmos genéticos pueden ser aplicados de forma exitosa en problemas donde 
necesitemos optimizar algo. En nuestro caso, queremos optimizar el trazado de una línea a que 
respete la forma original sobre la que se está dibujando.

Finalmente el último paso se realizará de forma trivial por nosotros mismos, 
exportando la imagen a SVG.

\bibliography{memoria}
\bibliographystyle{acm}
\nocite{*}

\end{document}
